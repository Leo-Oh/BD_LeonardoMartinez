\documentclass[10pt]{article}         %% What type of document you're writing.
\usepackage{graphicx}
\usepackage{hyperref}
\usepackage[dvipsnames]{xcolor}

%%%%% Preamble

%% Packages to use

\usepackage{amsmath,amsfonts,amssymb}   %% AMS mathematics macros

%% Title Information.

\title{University Case Data Model}
\author{Leonardo Martinez}
%% \date{29 sep 2020}           %% By default, LaTeX uses the current date

%%%%% The Document

\begin{document}

\maketitle

\begin{abstract}
This document implements the University Case Data Model.
\end{abstract}

\section{Data Model Desciption}


\textcolor{red}{Alumnos} con ( \textcolor{green}{matricula, nombre, direccion, carrera [sistemas, nanotecnologia, fisico nuclear, matematicas, biotecnologia]} )\\

\textcolor{red}{Materias} con ( \textcolor{green}{codigo, nombre, creditos} )\\

\textcolor{red}{Profesores } con ( \textcolor{green}{nomina, nombre completo, grado\_estudios [lic, maestria, doctorado, postdoctorado]} )\\



Los alumnos pueden \textcolor{yellow}{tomar} n materias, y 
\\
una materia es \textcolor{yellow}{tomada} por n alumnos, es importante guardar el semestre [1, 2, 3, 4,5, 6, 7, 8] y la calificacion en esta relacion
\\
Los profesores \textcolor{yellow}{imparten} n materias y una materia tambien puede ser \textcolor{yellow}{impartida} por n profesores



\section{E-R Model}

University Case ...
\begin{figure}[h]
     \includegraphics[scale=0.3]{er_university.png}
     \caption{University Case E-R Model}
\end{figure}

\section{Relation Model}

University Case ...
\begin{figure}[h]
     \includegraphics[scale=0.3]{relation.png}
     \caption{University Relation Model}
\end{figure}


\end{document}
