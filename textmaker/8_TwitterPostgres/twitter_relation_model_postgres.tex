\documentclass[10pt]{article}         %% What type of document you're writing.
\usepackage{graphicx}
\usepackage{hyperref}
\usepackage[dvipsnames]{xcolor}

%%%%% Preamble

%% Packages to use

\usepackage{amsmath,amsfonts,amssymb}   %% AMS mathematics macros

%% Title Information.

\title{Twitter Data Model}
\author{Leonardo Martinez}
%% \date{29 sep 2020}           %% By default, LaTeX uses the current date

%%%%% The Document

\begin{document}

\maketitle

\begin{abstract}
This document implements the Twitter Data Model.
\end{abstract}

\section{Data Model Desciption}


\textcolor{red}{Usuarios} de twitter ( \textcolor{green}{idusuario,usuario,email,telefono,nombre} )\\

\textcolor{red}{tweets} ( \textcolor{green}{tweet,urlimagen} )\\

un usuario \textcolor{yellow}{escribe} tweets

\section{E-R Model}

Twitter ...
\begin{figure}[h]
     \includegraphics[scale=0.5]{twitter_model}
     \caption{Twitter E-R Model}
\end{figure}

\section{Build Database in postgresql}
\begin{enumerate}
	\item
	ssh -i leonardodmm leonardodmm@104.198.244.0
	\item
	sudo -u postgres createdb leonardodmm\_twitter;
	\item
	sudo -postgres psql;
	\item
	\textbackslash list;
	\item
	connect leonardodmm\_twitter
	\item
	\textbackslash dt;
	\item
	create table usuarios(idUsuario varchar(50),usuario varchar(20),\\email varchar(40),passwd varchar(8),telefono varchar(10),nombres varchar(20), apellidoPat varchar(20), apellidoMat varchar (20));
	\item
	select * from usuarios;
	\item
	alter table usuarios add constraint pk\_idusuario primary key (idusuario);
	\item
	insert into usuarios values('1','leonardo','leonardo.m2349@gmail.com','qwerty','2711426743',\\'leonardo',',montiel','martinez');
	
\end{enumerate}

\end{document}
