\documentclass[10pt]{article}         %% What type of document you're writing.
\usepackage{graphicx}
\usepackage{hyperref}
\usepackage[dvipsnames]{xcolor}

%%%%% Preamble

%% Packages to use

\usepackage{amsmath,amsfonts,amssymb}   %% AMS mathematics macros

%% Title Information.

\title{CASO FACTURAS Data Model}
\author{Leonardo Martinez}
%% \date{29 sep 2020}           %% By default, LaTeX uses the current date

%%%%% The Document

\begin{document}

\maketitle

\begin{abstract}
This document implements the CASO FACTURAS Data Model.
\end{abstract}

\section{Data Model Desciption}


\textcolor{red}{Cliente} de CASO FACTURAS ( \textcolor{green}{ rfc, nombre y direccion} )\\

\textcolor{red}{facturas} ( \textcolor{green}{folio, fecha, concepto, subtotal, iva y total} )\\

un cliente  \textcolor{yellow}{solicita n} facturas

\section{E-R Model}

CASO FACTURAS ...
\begin{figure}[h]
     \includegraphics[scale=0.5]{casoFacturas.png}
     \caption{CASO FACTURAS E-R Model}
\end{figure}



\begin{figure}[h]
     \includegraphics[scale=0.5]{casoFacturasRelation.png}
     \caption{CASO FACTURAS Relation}
\end{figure}

\end{document}
