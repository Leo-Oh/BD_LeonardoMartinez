\documentclass[10pt]{article}         %% What type of document you're writing.
\usepackage{graphicx}
\usepackage{hyperref}
\usepackage[dvipsnames]{xcolor}

%%%%% Preamble

%% Packages to use

\usepackage{amsmath,amsfonts,amssymb}   %% AMS mathematics macros

%% Title Information.

\title{INEGI Data Model}
\author{Leonardo Martinez}
%% \date{1 octubre 2020}           %% By default, LaTeX uses the current date

%%%%% The Document

\begin{document}

\maketitle

\begin{abstract}
This document implements the INEGI Data Model.
\end{abstract}

\section{Data Model Desciption}


\textcolor{red}{Entidades} de mexico ( \textcolor{green}{idEntidad,nombreEntidad} )\\
\textcolor{red}{Municipios} (\textcolor{green}{idMunicipio,nombreMunicipio})\\
\textcolor{red}{Empresa} (\textcolor{green}{idEmpres,nombreEmpresa,domicilio,tipoActividad [hospital, escuela, oxxo, gobierno, cajero...], latitud, longitud})\\

\begin{enumerate}
\item
Las Entidades se \textcolor{yellow}{componen} de Municipios \\
\item
Los Municipios \textcolor{yellow}{tienen} Empresas\\
\end{enumerate}


\section{E-R Model}

INEGI...

\begin{figure}[h]
     \includegraphics[scale=0.6]{er_INEGI_model}
     \caption{INEGI E-R Model}
\end{figure}
   

\end{document}

