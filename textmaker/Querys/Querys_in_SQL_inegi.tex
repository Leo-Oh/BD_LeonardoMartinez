\documentclass[10pt]{article}         %% What type of document you're writing.
\usepackage{graphicx}
\usepackage{hyperref}
\usepackage[dvipsnames]{xcolor}

%%%%% Preamble

%% Packages to use

\usepackage{amsmath,amsfonts,amssymb}   %% AMS mathematics macros

%% Title Information.

\title{INEGI Data Model}
\author{Leonardo Martinez}
%% \date{1 octubre 2020}           %% By default, LaTeX uses the current date

%%%%% The Document

\begin{document}

\maketitle

\begin{abstract}
This document implements the INEGI Data Model.
\end{abstract}

\section{Data Model Desciption}


\textcolor{red}{Entidades} de mexico ( \textcolor{green}{idEntidad,nombreEntidad} )\\
\textcolor{red}{Municipios} (\textcolor{green}{idMunicipio,nombreMunicipio})\\
\textcolor{red}{Empresa} (\textcolor{green}{idEmpres,nombreEmpresa,domicilio,tipoActividad [hospital, escuela, oxxo, gobierno, cajero...], latitud, longitud})\\

\begin{enumerate}
\item
Las Entidades se \textcolor{yellow}{componen} de Municipios \\
\item
Los Municipios \textcolor{yellow}{tienen} Empresas\\
\end{enumerate}


\section{E-R Model}

INEGI...

\begin{figure}[b]
     \includegraphics[scale=0.6]{er_INEGI_model}
     \caption{INEGI E-R Model}
\end{figure}
   
\section{Relational Model}
INEGI Relational Model

\begin{figure}[b]
     \includegraphics[scale=0.4]{relational_Inegi_model2}
     \caption{INEGI Relational Model}
\end{figure}

\section{Steps}
\begin{enumerate}

\item
	sudo -u postgres createdb leonardodmm\_inegi;
\item
	sudo -u postgres psql;
\item
	\textbackslash connect leonardodmm\_inegi;
\item
	create table entidades(identidad int,nombreentidad varchar(200));
\item
	alter table entidades add constraint pk\_identidad primary key(identidad);
\item
	insert into entidades values (1,'AGUASCALIENTES');\\
	insert into entidades values (2,'BAJA CALIFORNIA');\\
	insert into entidades values (2,'BAJA CALIFORNIA SUR');
\item
	create table municipios(idmunicipios int,identidad int, nombremunicipio varchar(200));
\item
	alter table municipios add constraint pk\_identidad\_idmunicipio primary key(identidad,idmunicipio);
\item
	alter table municipios add constraint fk\_identidad foreign key(identidad) references entidades(identidad);
\item
	insert into municipios values(1,1 'EL LLANO');
	insert into municipios values(1,2 'TIJUANA');
	insert into municipios values(2,2 'MEXICALI');
\item
	create table tipoactividad(codigoactividad int,descripcion varchar (200));
\item
	alter table tipoactividad add contraint pk\_codigoactividad primary key(codigoactividad);
\item	
	 insert into tipoactividad values(522110.'BANCA MULTIPLE');\\
	 insert into tipoactividad values(522451.'MONTEPIOS');
\item
	create table empresas(idempresa int,identidad int,idmunicipio int,codigoactividad int,nombreempresa varchar(200),latitud float,longitud float,calle varchar(100),numero int,colonia varchar(100),codigopostal int,ciudad varchar(100),estado varchar(50),pais varchar(50));
\item
	alter table empresa add constraint pk\_id\_empresa\_identidad\_idmunicipio primary key(idempresa, identidad,idmunicipio);
\item
	alter table empresas add constraint pk\_id\_empresa\_identidad\_idmunicipio doreign key(identidad,idmunicipio) references municipios(identidad, idmunicipio);
\item
alter table empresas add constraint fk\_codigoactividad foreign key(codigoactividad) references tipoactividad(codigoactividad);
\item
insert into empresas values (1, 1, 1, 522110,'SUCURSAL BANAMEX 1',21.88234,-102.28259,'AV 1',1,'CENTRO',98800,'AGUASCALIENTES','AGS','MEXICO');
\\
insert into empresas values (3, 1, 1, 522451,'BANCOMER 1',21.88255,-102.28259,'AV 1',1,'CENTRO',98800,'AGUASCALIENTES','AGS','MEXICO');

\end{enumerate}
\section{Querys in SQL}
\begin{enumerate}
	\item proyeccion: select field1, field2 ...\\
		select identidad, nombreentidad from entidades;\\
		\\
		select nombre entidad from entidades;\\
	
	\item proyeccion con alias:\\
		select identidad as id, nombreentidad as estado from entidades;\\
		\\
		select identidad, nombreentidad as estado from entidades;
		
	\item seleccion:\\
		select identidad, nombreentidad as estado from entidades where identidad = 1;\\
		\\
		select identidad, nombreentidad as estado from entidades where identidad > 1;\\
		\\
		select identidad, nombreentidad as estado from entidades where identidad <= 2;\\
		
	\item seleccion  y proyeccion
		select nombreentidad as estado from entidades where identidad <=1;
		
	\item Seleccion , proyeccion y alias:\\
		select nombreentidad as estado from entidades where identidad <=2;
		
	\item seleccion,proyeccion, alias y rango:\\
		select nombreentidad as estado from entidades where identidad in (2,3);
		
	\item seleecion con operadores logicos:\\
		select * from empresas where identidad = 1 and idempresas = 1;\\
		\\
		select idempresa, identidad, idmunicipio, nombreempresa from empresas where idempresa = 1 or 		idempresa = 2;\\
		\\
		select idempresa, identidad, idmunicipio, nombre, codigoactividad from empresas where identidad = 1 and codigoactividad in(522110);
	
	\item count:\\
		select count(*) from empresas;\\
		\\
		select count(*) as numempresas from empresas;\\
		\\
		select count(*) as numempresas from empresas where identidad = 1;

	\item avg:\\
		select avg(latitud) from empresas where identidad = 1;

	\item sum:\\
		select sum(codigoactividad) from empresas where identidad = 1;

	\item min:\\
		select min(codigoactividad) from empresas where identidad = 1;

	\item max:\\
		select max(codigoactividad) from empresas where identidad = 1;

	\item stadistic functions:\\
		select count(*), sum(codigoactividad), av(codigoactividad), max(codigoactividad),
min(codigoactividad) from empresas where identidad = 1;

	\item PRODUCTO CARTESIANO 2X2:\\
		select * from entidades, municipios;\\
		\\
		select * from entidades, municipios where entidades.identidad = municipios.identidad;\\
		\\
		select entidades, identidad, nombreentidad, idmunicipio, nombremunicipio from entidades, municipio where entidades. identidad = municipios.identidad;\\
		\\
		select entidades.identidad,nombreentidad, idmunicipio,nombremunicipio from entidades,municipios where entidades.identidad = municipios.identidad and entidades.identidad in (2,3);
	
	\item PRODUCTO CARTESIANO 3X3:\\
		select * from entidades, municipios, empresas where entidades.identidad = municipios.identidad and (municipios.identidad = empresas.identidad and municipios.idmunicipio = empresas.idmunicipio);\\
\\
select nombreentidad, nombremunicipio,nombreempresa,latitud,longitud from entidades,municipios,empresas where entidades.identidad = municipios.identidad and (municipios.identidad = empresas.identidad and municipios.idmunicipio = empresas.idmunicipio);

\end{enumerate}

\end{document}

