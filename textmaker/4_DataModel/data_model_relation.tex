\documentclass[10pt]{article}         %% What type of document you're writing.
\usepackage{graphicx}
\usepackage{hyperref}
\usepackage[dvipsnames]{xcolor}

%%%%% Preamble

%% Packages to use

\usepackage{amsmath,amsfonts,amssymb}   %% AMS mathematics macros

%% Title Information.

\title{Data Model}
\author{Leonardo Martinez}
%% \date{29 sep 2020}           %% By default, LaTeX uses the current date

%%%%% The Document

\begin{document}

\maketitle

\begin{abstract}
This document implements the Data Model.
\end{abstract}

\section{Data Model Desciption}

\begin{enumerate}

\item
The company is organized into departmen.\\
Each deparment has a unique \textcolor{green}{name}, a unique \textcolor{green}{number}\\
and a particular employee who \textcolor{yellow}{manages} the department.
we keep track of the \textcolor{green}{start date}when that employee began mangin the deparment\\
A deparment may have several \textcolor{green}{locations}

\item
A deparment \textcolor{yellow}{controls} a number of projects, each of whicj has a unique \textcolor{green}{name}, a unique \textcolor{green}{number}, and a single \textcolor{green}{locations}

\item
The database will store each employee’s \textcolor{green}{name},\textcolor{green}{Social Security number},\textcolor{green}{
address}, \textcolor{green}{salary}, \textcolor{green}{sex (gender)}, and \textcolor{green}{birth date}. 
\\
An employee \textcolor{yellow}{is assigned} to one
department, but may \textcolor{green}{work} on several projects, which are not necessarily
controlled by the same department. It is required to keep track of the cur-
rent \textcolor{green}{number of hours per week} that an employee works on each project
\\
***as well as the direct \textcolor{yellow}{supervisor} of each employee (who is another employee)***.

\end{enumerate}


\section{E-R Model}

Data Model ...
\begin{figure}[h]
     \includegraphics[scale=0.4]{datamodel}
     \caption{data model E-R Model}
\end{figure}

\section{Relational Model}
Data Model  Relational Model

 \begin{figure}[h]
     \includegraphics[scale=0.4]{datamodel2}
     \caption{data model Relation Model}
\end{figure}


\end{document}
