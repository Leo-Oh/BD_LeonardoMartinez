\documentclass[10pt]{article}         %% What type of document you're writing.
\usepackage{graphicx}
\usepackage{hyperref}
\usepackage[dvipsnames]{xcolor}

%%%%% Preamble

%% Packages to use

\usepackage{amsmath,amsfonts,amssymb}   %% AMS mathematics macros

%% Title Information.

\title{Twitter Data Model}
\author{Leonardo Martinez}
%% \date{29 sep 2020}           %% By default, LaTeX uses the current date

%%%%% The Document

\begin{document}

\maketitle

\begin{abstract}
This document implements the Twitter Data Model.
\end{abstract}

\section{Data Model Desciption}


\textcolor{red}{Usuarios} de twitter ( \textcolor{green}{idusuario,usuario,email,telefono,nombre} )\\

\textcolor{red}{tweets} ( \textcolor{green}{tweet,urlimagen} )\\

un usuario \textcolor{yellow}{escribe} tweets

\section{E-R Model}

Twitter ...

\begin{figure}[h]
     \includegraphics[scale=0.6]{twitter_model}
     \caption{Twitter Model}
\end{figure}
   
\section{Relational Model}
Twitter Relational Model



\end{document}
